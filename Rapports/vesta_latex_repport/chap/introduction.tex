\section{Introduction}

This semester project takes part of the CHIC2015 which is a new project initiated at EPFL by Marc Laperrouza. The goal of this new project is for students develop a project from idea to production in a semester. Indeed a trip to Shenzhen in China is programmed following the end of the semester(july). The prototype will be produced by chinese company called Seeedstudio. They mainly do pcb prototyping but they also mill and 3d print parts.

Three groups of 5 people are created after a brainstorming week-end at the beggining of the semester with for each group one HEC (economy) student, one ECAL (industrial design) student and three engineering students from EPFL. For the project presented in this report, two engineers are from microengineering and one is from material science.

The economy student works on the buisness part of the project which contains the buisness model, the market definition and the value proposition. The industrial design student works on the mecanical and software design. The engineering students work on the materials, hardware and software of the project.

After the brainstorming week-end, the rough idea was to connect elderly people to their families with an easy to use electronic device. Many elderly people are excluded from the new technologies and are therefore excluded from modern communications.

In this report, the focus will be put on the engineering part of the project especially. An abstract to the market researches will be introduced. The industrial design and the material parts will also be discussed to have a global aspect of the project.
