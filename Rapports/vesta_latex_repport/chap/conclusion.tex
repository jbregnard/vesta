\section{Conclusion}

The software is not finished yet but the hardware and industrial design are already well implemented. The priority was placed on the hardware and the design. These two parts must be finished before going to China but the software can be finished afterwards.

This device can be really interesting for different projects that need a screen. As the beaglebone black has unused GPIOs it is possible to connect some sensors or other devices to it. Bluetooth is not used at the moment but can be really interesting to have it for some applications.

The first iteration of the PCB will be made this summer but different tests were done with the hardware. The touchscreen is fully recognized by the OS and works perfectly. The accelerometer also works. The WIFI/bluetooth chip couldn't be tested yet because no development kit was available for the beaglebone black at the moment (a cape is made by boardzoo but will be available in the middle of june only). This chip shouldn't cause any problem but the critical step will be its recognition by the linux OS.

For the industrial design, the case will be 3d printed but every part is compatible with the plastic molding production technic. The industrial designer worked with the engineers for the assembly to be sure that everything fits in the case.

This project was really interesting and instructive. Working in a team with students from different schools and sections is something new and corresponds to the real work of an engineer in a company. It was sometimes difficult to communicate with the same words as everybody has a different background. The timetable was quite short for the complexity of the project but it was very stimulating. Hopefully we will come back this summer with a fully functionnal prototype. 
