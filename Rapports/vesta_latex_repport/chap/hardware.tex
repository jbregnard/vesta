\section{Hardware}
This section is dedicated to show all the electronic components we chose and how they work together. As described earlier the PCB containing all the components will be manufactured by Seeedstudio. They manufacture PCBs with other hardware components and they sell quite a number of different breakout boards and other developpement parts. A request made by CHIC managing team was to source as many components as possible at Seeedstudio. Of course not all components we needed were available by them so some of the components are from different sources. 

Hardware :
\begin{itemize}
\item{Beaglebone black, will be further mentioned as BBB}
\item{SD card (containing the Operating System)}
\item{PCB}
\item{Wl1835mod WIFI and bluetooth module}
\item{Capacitive touch screen}
\item{ADXL-345 accelerometer}
\item{Some other tension regulators and a battery drive}
\end{itemize}

\subsection{Beaglebone Black}
The BBB is an open-hardware microcomputer very famous for development. It uses an ARM processor from Texas Instruments, the AM335X. It is clocked at 1GHz.
\begin{table}[!htbp]
  \begin{center}
    \begin{tabular}{|l|r|r|}%p{5cm}|}
      \hline
        measurement N$^{\circ}$ & CW [mNm] & CCW [mNm]\\ \hline %\hline
        1 & 2.24e-4 & -6.32e-4 \\ \hline
        2 & 2.073-4 & -6.39e-4\\ \hline
        3 & 2.72e-4 & -6.64e-4\\ \hline
        4 & 2.46e-4 & -6.46e-4\\ \hline
        5 & 2.72e.4 & -6.64e-4\\ \hline \hline
        mean & 2.44e-4 & -6.49e-4\\ 
         \hline
    \end{tabular}
  \end{center}
  \caption {Dry friction torque measurement} \label{tab:dry friction meas} 
\end{table}
\subsection{Screen and Touchscreen}
We first ordered a resistive touchscreen BBB cape available from seeedstudio to see how its was made and to see if the resistive technology was applicable to our project. We quickly realised that the resistive tuchscreen was not very adapted to manipulate photos. Especially the very well known “swipe” geasture to move from one photo to another was impossible to do with the resistive touchscreen.
So we looked for a similar screen but one using a capacitive touchscreen. We started by looking on chinese sites which had the cheapest screens of course. But the cheap chinese screens were lacking documention. At the time we were unfamiliar with the interface the screens used and we needed at least a datasheet to get going.
 Finally we chose a screen we found on Mouser which had good documentation and especially there was an existing driver for the touchscreen IC in the linux kernel we were going to use.
\subsubsection{Screen}
At the current state of our prototype the main function is to view images 
Table with screen specs.
\subsubsection{Touchscreen}
\subsubsection{Backlight}
\subsection{Wireless Communication}
\subsubsection{Wlan}
\subsubsection{Bluetooth}
\subsubsection{Power management}
\subsection{Front LED}
The LED is there to inform the user of a new message. Therefor it could be quite low power. Seeedstudio only had an RGB LED which we ordered although for the final led we will use is a warm white single color LED. 

We want it to flash in a heartbeat patern. This is achieved by using a PWM pin of the BBB
\subsection{Power management}
\subsection{Batteries and charger}
\subsection{PCB}
\subsection{Cost estimates}
